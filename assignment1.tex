% Options for packages loaded elsewhere
\PassOptionsToPackage{unicode}{hyperref}
\PassOptionsToPackage{hyphens}{url}
%
\documentclass[
]{article}
\usepackage{amsmath,amssymb}
\usepackage{iftex}
\ifPDFTeX
  \usepackage[T1]{fontenc}
  \usepackage[utf8]{inputenc}
  \usepackage{textcomp} % provide euro and other symbols
\else % if luatex or xetex
  \usepackage{unicode-math} % this also loads fontspec
  \defaultfontfeatures{Scale=MatchLowercase}
  \defaultfontfeatures[\rmfamily]{Ligatures=TeX,Scale=1}
\fi
\usepackage{lmodern}
\ifPDFTeX\else
  % xetex/luatex font selection
\fi
% Use upquote if available, for straight quotes in verbatim environments
\IfFileExists{upquote.sty}{\usepackage{upquote}}{}
\IfFileExists{microtype.sty}{% use microtype if available
  \usepackage[]{microtype}
  \UseMicrotypeSet[protrusion]{basicmath} % disable protrusion for tt fonts
}{}
\makeatletter
\@ifundefined{KOMAClassName}{% if non-KOMA class
  \IfFileExists{parskip.sty}{%
    \usepackage{parskip}
  }{% else
    \setlength{\parindent}{0pt}
    \setlength{\parskip}{6pt plus 2pt minus 1pt}}
}{% if KOMA class
  \KOMAoptions{parskip=half}}
\makeatother
\usepackage{xcolor}
\usepackage[margin=1in]{geometry}
\usepackage{color}
\usepackage{fancyvrb}
\newcommand{\VerbBar}{|}
\newcommand{\VERB}{\Verb[commandchars=\\\{\}]}
\DefineVerbatimEnvironment{Highlighting}{Verbatim}{commandchars=\\\{\}}
% Add ',fontsize=\small' for more characters per line
\usepackage{framed}
\definecolor{shadecolor}{RGB}{248,248,248}
\newenvironment{Shaded}{\begin{snugshade}}{\end{snugshade}}
\newcommand{\AlertTok}[1]{\textcolor[rgb]{0.94,0.16,0.16}{#1}}
\newcommand{\AnnotationTok}[1]{\textcolor[rgb]{0.56,0.35,0.01}{\textbf{\textit{#1}}}}
\newcommand{\AttributeTok}[1]{\textcolor[rgb]{0.13,0.29,0.53}{#1}}
\newcommand{\BaseNTok}[1]{\textcolor[rgb]{0.00,0.00,0.81}{#1}}
\newcommand{\BuiltInTok}[1]{#1}
\newcommand{\CharTok}[1]{\textcolor[rgb]{0.31,0.60,0.02}{#1}}
\newcommand{\CommentTok}[1]{\textcolor[rgb]{0.56,0.35,0.01}{\textit{#1}}}
\newcommand{\CommentVarTok}[1]{\textcolor[rgb]{0.56,0.35,0.01}{\textbf{\textit{#1}}}}
\newcommand{\ConstantTok}[1]{\textcolor[rgb]{0.56,0.35,0.01}{#1}}
\newcommand{\ControlFlowTok}[1]{\textcolor[rgb]{0.13,0.29,0.53}{\textbf{#1}}}
\newcommand{\DataTypeTok}[1]{\textcolor[rgb]{0.13,0.29,0.53}{#1}}
\newcommand{\DecValTok}[1]{\textcolor[rgb]{0.00,0.00,0.81}{#1}}
\newcommand{\DocumentationTok}[1]{\textcolor[rgb]{0.56,0.35,0.01}{\textbf{\textit{#1}}}}
\newcommand{\ErrorTok}[1]{\textcolor[rgb]{0.64,0.00,0.00}{\textbf{#1}}}
\newcommand{\ExtensionTok}[1]{#1}
\newcommand{\FloatTok}[1]{\textcolor[rgb]{0.00,0.00,0.81}{#1}}
\newcommand{\FunctionTok}[1]{\textcolor[rgb]{0.13,0.29,0.53}{\textbf{#1}}}
\newcommand{\ImportTok}[1]{#1}
\newcommand{\InformationTok}[1]{\textcolor[rgb]{0.56,0.35,0.01}{\textbf{\textit{#1}}}}
\newcommand{\KeywordTok}[1]{\textcolor[rgb]{0.13,0.29,0.53}{\textbf{#1}}}
\newcommand{\NormalTok}[1]{#1}
\newcommand{\OperatorTok}[1]{\textcolor[rgb]{0.81,0.36,0.00}{\textbf{#1}}}
\newcommand{\OtherTok}[1]{\textcolor[rgb]{0.56,0.35,0.01}{#1}}
\newcommand{\PreprocessorTok}[1]{\textcolor[rgb]{0.56,0.35,0.01}{\textit{#1}}}
\newcommand{\RegionMarkerTok}[1]{#1}
\newcommand{\SpecialCharTok}[1]{\textcolor[rgb]{0.81,0.36,0.00}{\textbf{#1}}}
\newcommand{\SpecialStringTok}[1]{\textcolor[rgb]{0.31,0.60,0.02}{#1}}
\newcommand{\StringTok}[1]{\textcolor[rgb]{0.31,0.60,0.02}{#1}}
\newcommand{\VariableTok}[1]{\textcolor[rgb]{0.00,0.00,0.00}{#1}}
\newcommand{\VerbatimStringTok}[1]{\textcolor[rgb]{0.31,0.60,0.02}{#1}}
\newcommand{\WarningTok}[1]{\textcolor[rgb]{0.56,0.35,0.01}{\textbf{\textit{#1}}}}
\usepackage{graphicx}
\makeatletter
\def\maxwidth{\ifdim\Gin@nat@width>\linewidth\linewidth\else\Gin@nat@width\fi}
\def\maxheight{\ifdim\Gin@nat@height>\textheight\textheight\else\Gin@nat@height\fi}
\makeatother
% Scale images if necessary, so that they will not overflow the page
% margins by default, and it is still possible to overwrite the defaults
% using explicit options in \includegraphics[width, height, ...]{}
\setkeys{Gin}{width=\maxwidth,height=\maxheight,keepaspectratio}
% Set default figure placement to htbp
\makeatletter
\def\fps@figure{htbp}
\makeatother
\setlength{\emergencystretch}{3em} % prevent overfull lines
\providecommand{\tightlist}{%
  \setlength{\itemsep}{0pt}\setlength{\parskip}{0pt}}
\setcounter{secnumdepth}{-\maxdimen} % remove section numbering
\ifLuaTeX
  \usepackage{selnolig}  % disable illegal ligatures
\fi
\usepackage{bookmark}
\IfFileExists{xurl.sty}{\usepackage{xurl}}{} % add URL line breaks if available
\urlstyle{same}
\hypersetup{
  pdftitle={Assigment1},
  hidelinks,
  pdfcreator={LaTeX via pandoc}}

\title{Assigment1}
\author{}
\date{\vspace{-2.5em}2024-10-02}

\begin{document}
\maketitle

\begin{Shaded}
\begin{Highlighting}[]
\CommentTok{\#Loading needed packages}

\FunctionTok{library}\NormalTok{(tidyverse)}
\NormalTok{conflicted}\SpecialCharTok{::}\FunctionTok{conflict\_prefer}\NormalTok{(}\StringTok{"filter"}\NormalTok{, }\StringTok{"dplyr"}\NormalTok{)}
\end{Highlighting}
\end{Shaded}

\begin{verbatim}
## [conflicted] Removing existing preference.
## [conflicted] Will prefer dplyr::filter over any other package.
\end{verbatim}

\begin{Shaded}
\begin{Highlighting}[]
\FunctionTok{library}\NormalTok{(viridis)}
\FunctionTok{library}\NormalTok{(}\StringTok{"vegan"}\NormalTok{)}
\FunctionTok{library}\NormalTok{(ggplot2)}
\FunctionTok{library}\NormalTok{(iNEXT)}

\CommentTok{\#I opened the Nematode data file from bold and explored the variables in the data set }
\NormalTok{dfNematodes }\OtherTok{\textless{}{-}} \FunctionTok{read\_tsv}\NormalTok{(}\AttributeTok{file =} \StringTok{"../data/Nematode\_data.tsv"}\NormalTok{)}
\end{Highlighting}
\end{Shaded}

\begin{verbatim}
## Warning: One or more parsing issues, call `problems()` on your data frame for details, e.g.:
##   dat <- vroom(...)
##   problems(dat)
\end{verbatim}

\begin{verbatim}
## Rows: 55686 Columns: 68
## -- Column specification ----------------------------------------------------------------------------------
## Delimiter: "\t"
## chr (46): processid, sampleid, catalognum, fieldnum, institution_storing, bin_uri, phylum_name, class_...
## dbl (13): recordID, phylum_taxID, class_taxID, order_taxID, family_taxID, subfamily_taxID, genus_taxID...
## lgl  (9): collection_code, tax_note, collection_event_id, collectiondate_start, collectiondate_end, co...
## 
## i Use `spec()` to retrieve the full column specification for this data.
## i Specify the column types or set `show_col_types = FALSE` to quiet this message.
\end{verbatim}

\begin{Shaded}
\begin{Highlighting}[]
\FunctionTok{names}\NormalTok{(dfNematodes)}
\end{Highlighting}
\end{Shaded}

\begin{verbatim}
##  [1] "processid"                  "sampleid"                   "recordID"                  
##  [4] "catalognum"                 "fieldnum"                   "institution_storing"       
##  [7] "collection_code"            "bin_uri"                    "phylum_taxID"              
## [10] "phylum_name"                "class_taxID"                "class_name"                
## [13] "order_taxID"                "order_name"                 "family_taxID"              
## [16] "family_name"                "subfamily_taxID"            "subfamily_name"            
## [19] "genus_taxID"                "genus_name"                 "species_taxID"             
## [22] "species_name"               "subspecies_taxID"           "subspecies_name"           
## [25] "identification_provided_by" "identification_method"      "identification_reference"  
## [28] "tax_note"                   "voucher_status"             "tissue_type"               
## [31] "collection_event_id"        "collectors"                 "collectiondate_start"      
## [34] "collectiondate_end"         "collectiontime"             "collection_note"           
## [37] "site_code"                  "sampling_protocol"          "lifestage"                 
## [40] "sex"                        "reproduction"               "habitat"                   
## [43] "associated_specimens"       "associated_taxa"            "extrainfo"                 
## [46] "notes"                      "lat"                        "lon"                       
## [49] "coord_source"               "coord_accuracy"             "elev"                      
## [52] "depth"                      "elev_accuracy"              "depth_accuracy"            
## [55] "country"                    "province_state"             "region"                    
## [58] "sector"                     "exactsite"                  "image_ids"                 
## [61] "image_urls"                 "media_descriptors"          "captions"                  
## [64] "copyright_holders"          "copyright_years"            "copyright_licenses"        
## [67] "copyright_institutions"     "photographers"
\end{verbatim}

\begin{Shaded}
\begin{Highlighting}[]
\CommentTok{\# I made a new table containing only the variables I needed}
\NormalTok{dfNematodes.sub }\OtherTok{\textless{}{-}}\NormalTok{ dfNematodes[, }\FunctionTok{c}\NormalTok{(}\StringTok{"processid"}\NormalTok{, }\StringTok{"bin\_uri"}\NormalTok{,}\StringTok{"family\_name"}\NormalTok{,}\StringTok{"genus\_name"}\NormalTok{, }\StringTok{"species\_name"}\NormalTok{, }\StringTok{"country"}\NormalTok{, }\StringTok{"lat"}\NormalTok{, }\StringTok{"lon"}\NormalTok{)]}

\CommentTok{\#I made a table with just the families to make sure Onchocercidae was sampled}
\NormalTok{Nematodes.Families }\OtherTok{\textless{}{-}} \FunctionTok{table}\NormalTok{(dfNematodes.sub}\SpecialCharTok{$}\NormalTok{family\_name)}

\CommentTok{\#Then I made sure there was enough Onchocercidae sampled}
\FunctionTok{names}\NormalTok{(Nematodes.Families)}
\end{Highlighting}
\end{Shaded}

\begin{verbatim}
##   [1] "Actinolaimidae"              "Acuariidae"                  "Agfidae"                    
##   [4] "Alaimidae"                   "Allantonematidae"            "Alloionematidae"            
##   [7] "Amphibiophilidae"            "Anatonchidae"                "Ancylostomatidae"           
##  [10] "Angiostomatidae"             "Anguillicolidae"             "Anguinidae"                 
##  [13] "Anisakidae"                  "Anoplostomatidae"            "Anticomidae"                
##  [16] "Aphelenchidae"               "Aphelenchoididae"            "Aporcelaimidae"             
##  [19] "Ascarididae"                 "Ascaridiidae"                "Aspidoderidae"              
##  [22] "Axonolaimidae"               "Bastianiidae"                "Belondiridae"               
##  [25] "Belonolaimidae"              "Boleodoridae"                "Brevibuccidae"              
##  [28] "Bunonematidae"               "Camacolaimidae"              "Camallanidae"               
##  [31] "Campydoridae"                "Capillariidae"               "Carcharolaimidae"           
##  [34] "Carnoyidae"                  "Cephalobidae"                "Chromadorida_incertae_sedis"
##  [37] "Chromadoridae"               "Comesomatidae"               "Cosmocercidae"              
##  [40] "Criconematidae"              "Cryptonchidae"               "Cucullanidae"               
##  [43] "Cyatholaimidae"              "Desmidocercidae"             "Desmodoridae"               
##  [46] "Desmoscolecidae"             "Diaphanocephalidae"          "Dioctophymatidae"           
##  [49] "Diphtherophoridae"           "Diplogastridae"              "Diplopeltidae"              
##  [52] "Diploscapteridae"            "Dolichodoridae"              "Dorylaimidae"               
##  [55] "Draconematidae"              "Dracunculidae"               "Enchelidiidae"              
##  [58] "Enoplidae"                   "Entaphelenchidae"            "Epsilonematidae"            
##  [61] "Ethmolaimidae"               "Filariidae"                  "Filaroididae"               
##  [64] "Gnathostomatidae"            "Gongylonematidae"            "Guyanemidae"                
##  [67] "Gyoeryiidae"                 "Habronematidae"              "Haliplectidae"              
##  [70] "Hedruridae"                  "Heligmosomidae"              "Hemicycliophoridae"         
##  [73] "Heterakidae"                 "Heterocheilidae"             "Heteroderidae"              
##  [76] "Heterorhabditidae"           "Heteroxynematidae"           "Hoplolaimidae"              
##  [79] "Iotonchidae"                 "Iotonchiidae"                "Ironidae"                   
##  [82] "Kathlaniidae"                "Lauratonematidae"            "Leptolaimidae"              
##  [85] "Leptonchidae"                "Leptosomatidae"              "Linhomoeidae"               
##  [88] "Longidoridae"                "Marimermithidae"             "Meloidogynidae"             
##  [91] "Mermithidae"                 "Mesorhabditidae"             "Metastrongylidae"           
##  [94] "Microlaimidae"               "Monhysteridae"               "Mononchidae"                
##  [97] "Monoposthiidae"              "Muspiceidae"                 "Mylonchulidae"              
## [100] "Myolaimidae"                 "Neotylenchidae"              "Nordiidae"                  
## [103] "Nygolaimidae"                "Odontolaimidae"              "Onchocercidae"              
## [106] "Oncholaimidae"               "Oxystominidae"               "Oxyuridae"                  
## [109] "Panagrolaimidae"             "Pandolaimidae"               "Pararhyssocolpidae"         
## [112] "Parasitylenchidae"           "Paratylenchidae"             "Phanodermatidae"            
## [115] "Pharyngodonidae"             "Philometridae"               "Physalopteridae"            
## [118] "Plectidae"                   "Pratylenchidae"              "Prismatolaimidae"           
## [121] "Pseudaliidae"                "Qudsianematidae"             "Quimperiidae"               
## [124] "Raphidascarididae"           "Rhabdiasidae"                "Rhabditidae"                
## [127] "Rhabditoididae"              "Rhabdochonidae"              "Rhabdodemaniidae"           
## [130] "Rhabdolaimidae"              "Rhigonematidae"              "Richtersiidae"              
## [133] "Rictulariidae"               "Robertdollfusidae"           "Rotylenchulidae"            
## [136] "Seinuridae"                  "Selachinematidae"            "Siphonolaimidae"            
## [139] "Skrjabillanidae"             "Soboliphymatidae"            "Sphaerolaimidae"            
## [142] "Sphaerulariidae"             "Spirocercidae"               "Spiruridae"                 
## [145] "Steinernematidae"            "Strongylidae"                "Strongyloididae"            
## [148] "Subuluridae"                 "Telotylenchidae"             "Tetrameridae"               
## [151] "Thelastomatidae"             "Thelaziidae"                 "Thoracostomopsidae"         
## [154] "Tobrilidae"                  "Travassosinematidae"         "Trefusiidae"                
## [157] "Trichinellidae"              "Trichodoridae"               "Trichosomoididae"           
## [160] "Trichostrongylidae"          "Trichuridae"                 "Tripylidae"                 
## [163] "Tripyloididae"               "Tylenchidae"                 "Tylencholaimidae"           
## [166] "Tylenchulidae"               "Xennellidae"                 "Xiphinematidae"             
## [169] "Xyalidae"
\end{verbatim}

\begin{Shaded}
\begin{Highlighting}[]
\FunctionTok{hist}\NormalTok{(}\AttributeTok{x =}\NormalTok{ Nematodes.Families, }\AttributeTok{xlab =} \StringTok{"Count of BOLD Records per Family"}\NormalTok{, }\AttributeTok{ylab =} \StringTok{"Frequency (No. Families"}\NormalTok{)}
\end{Highlighting}
\end{Shaded}

\includegraphics{C:/Users/sdwor/OneDrive - University of Guelph/1- MASTERS DEGREE/Software tools/assignment1_files/figure-latex/unnamed-chunk-1-1.pdf}

\begin{Shaded}
\begin{Highlighting}[]
\FunctionTok{sort}\NormalTok{(}\FunctionTok{table}\NormalTok{(dfNematodes.sub}\SpecialCharTok{$}\NormalTok{family\_name), }\AttributeTok{decreasing =} \ConstantTok{TRUE}\NormalTok{)[}\DecValTok{1}\SpecialCharTok{:}\DecValTok{10}\NormalTok{]}
\end{Highlighting}
\end{Shaded}

\begin{verbatim}
## 
##   Criconematidae   Pratylenchidae    Onchocercidae    Heteroderidae  Strongyloididae    Hoplolaimidae 
##             3766             2353             1932             1903             1675             1673 
##      Rhabditidae     Cephalobidae   Meloidogynidae Aphelenchoididae 
##             1451             1441             1387             1342
\end{verbatim}

\begin{Shaded}
\begin{Highlighting}[]
\FunctionTok{plot}\NormalTok{(}\FunctionTok{sort}\NormalTok{(}\FunctionTok{table}\NormalTok{(dfNematodes.sub}\SpecialCharTok{$}\NormalTok{family\_name), }\AttributeTok{decreasing =} \ConstantTok{TRUE}\NormalTok{)[}\DecValTok{1}\SpecialCharTok{:}\DecValTok{5}\NormalTok{])}
\end{Highlighting}
\end{Shaded}

\includegraphics{C:/Users/sdwor/OneDrive - University of Guelph/1- MASTERS DEGREE/Software tools/assignment1_files/figure-latex/unnamed-chunk-1-2.pdf}

\begin{Shaded}
\begin{Highlighting}[]
\CommentTok{\#I made a dataframe for just Onchocercidae}

\NormalTok{dfNematodes.Onchocercidae }\OtherTok{\textless{}{-}} \FunctionTok{subset}\NormalTok{(dfNematodes.sub, family\_name }\SpecialCharTok{==} \StringTok{"Onchocercidae"}\NormalTok{)}

\CommentTok{\#Then I looked at the number of bins vs species and cleaned up the data by removing missing points}

\FunctionTok{length}\NormalTok{(}\FunctionTok{unique}\NormalTok{(dfNematodes.Onchocercidae}\SpecialCharTok{$}\NormalTok{bin\_uri))}
\end{Highlighting}
\end{Shaded}

\begin{verbatim}
## [1] 91
\end{verbatim}

\begin{Shaded}
\begin{Highlighting}[]
\FunctionTok{length}\NormalTok{(}\FunctionTok{unique}\NormalTok{(dfNematodes.Onchocercidae}\SpecialCharTok{$}\NormalTok{species\_name))}
\end{Highlighting}
\end{Shaded}

\begin{verbatim}
## [1] 233
\end{verbatim}

\begin{Shaded}
\begin{Highlighting}[]
\NormalTok{dfNematodes.Onchocercidae}\SpecialCharTok{$}\NormalTok{spaces }\OtherTok{\textless{}{-}} \FunctionTok{str\_count}\NormalTok{(}\AttributeTok{string =}\NormalTok{ dfNematodes.Onchocercidae}\SpecialCharTok{$}\NormalTok{species\_name, }\AttributeTok{pattern =} \StringTok{"[}\SpecialCharTok{\textbackslash{}\textbackslash{}}\StringTok{s}\SpecialCharTok{\textbackslash{}\textbackslash{}}\StringTok{.}\SpecialCharTok{\textbackslash{}\textbackslash{}}\StringTok{d]"}\NormalTok{)}

\NormalTok{df.Onchocercidae}\OtherTok{\textless{}{-}} \FunctionTok{subset}\NormalTok{(dfNematodes.Onchocercidae, spaces }\SpecialCharTok{==} \DecValTok{1} \SpecialCharTok{\&} \FunctionTok{is.na}\NormalTok{(spaces) }\SpecialCharTok{==}\NormalTok{ F }\SpecialCharTok{\&} \FunctionTok{is.na}\NormalTok{(species\_name) }\SpecialCharTok{==}\NormalTok{ F }\SpecialCharTok{\&} \FunctionTok{is.na}\NormalTok{(bin\_uri) }\SpecialCharTok{==}\NormalTok{ F }\SpecialCharTok{\&} \FunctionTok{is.na}\NormalTok{(country) }\SpecialCharTok{==}\NormalTok{ F }\SpecialCharTok{\&} \FunctionTok{is.na}\NormalTok{(lat) }\SpecialCharTok{==}\NormalTok{ F)}

\CommentTok{\#Creating a scatter plot of the Sampling intensity and Species richness at different latitudes}

\NormalTok{ df.Speciesrichbylat }\OtherTok{\textless{}{-}}\NormalTok{ df.Onchocercidae }\SpecialCharTok{\%\textgreater{}\%}
   \FunctionTok{mutate}\NormalTok{(}\AttributeTok{lat\_groups =} \FunctionTok{round}\NormalTok{(lat)) }\SpecialCharTok{\%\textgreater{}\%}
   \FunctionTok{group\_by}\NormalTok{(lat\_groups) }\SpecialCharTok{\%\textgreater{}\%}
   \FunctionTok{summarize}\NormalTok{(}\FunctionTok{unique}\NormalTok{(species\_name))}\SpecialCharTok{\%\textgreater{}\%}
   \FunctionTok{summarize}\NormalTok{(}\AttributeTok{total =} \FunctionTok{n}\NormalTok{())}\SpecialCharTok{\%\textgreater{}\%}
   \FunctionTok{as.data.frame}\NormalTok{()}
\end{Highlighting}
\end{Shaded}

\begin{verbatim}
## Warning: Returning more (or less) than 1 row per `summarise()` group was deprecated in dplyr 1.1.0.
## i Please use `reframe()` instead.
## i When switching from `summarise()` to `reframe()`, remember that `reframe()` always returns an ungrouped
##   data frame and adjust accordingly.
## Call `lifecycle::last_lifecycle_warnings()` to see where this warning was generated.
\end{verbatim}

\begin{verbatim}
## `summarise()` has grouped output by 'lat_groups'. You can override using the `.groups` argument.
\end{verbatim}

\begin{Shaded}
\begin{Highlighting}[]
\NormalTok{ df.Samplingbylat }\OtherTok{\textless{}{-}}\NormalTok{ df.Onchocercidae }\SpecialCharTok{\%\textgreater{}\%}
   \FunctionTok{mutate}\NormalTok{(}\AttributeTok{lat\_groups =} \FunctionTok{round}\NormalTok{(lat)) }\SpecialCharTok{\%\textgreater{}\%}
   \FunctionTok{group\_by}\NormalTok{(lat\_groups) }\SpecialCharTok{\%\textgreater{}\%}
   \FunctionTok{summarize}\NormalTok{(}\AttributeTok{total =} \FunctionTok{n}\NormalTok{())}\SpecialCharTok{\%\textgreater{}\%}
   \FunctionTok{as.data.frame}\NormalTok{()}

\NormalTok{ df.graphbylat }\OtherTok{\textless{}{-}} \FunctionTok{bind\_rows}\NormalTok{(}\FunctionTok{list}\NormalTok{(}\StringTok{"Sampling Intensity"} \OtherTok{=}\NormalTok{ df.Samplingbylat,  }\StringTok{"Species richness"} \OtherTok{=}\NormalTok{ df.Speciesrichbylat), }\AttributeTok{.id =} \StringTok{"id"}\NormalTok{)}
 
 
\NormalTok{ gg.graphbylat }\OtherTok{\textless{}{-}} \FunctionTok{ggplot}\NormalTok{(df.graphbylat, }\FunctionTok{aes}\NormalTok{(}\AttributeTok{x=}\NormalTok{ lat\_groups, }\AttributeTok{y=}\NormalTok{ total))}\SpecialCharTok{+}
   \FunctionTok{geom\_point}\NormalTok{(}\AttributeTok{size =} \FloatTok{1.5}\NormalTok{)}\SpecialCharTok{+}
   \FunctionTok{geom\_smooth}\NormalTok{(}\AttributeTok{method =} \StringTok{"lm"}\NormalTok{, }\AttributeTok{color =} \FunctionTok{c}\NormalTok{(}\StringTok{"\#D55E00"}\NormalTok{))}\SpecialCharTok{+}
    \FunctionTok{theme\_bw}\NormalTok{(}\AttributeTok{base\_size =} \DecValTok{15}\NormalTok{)}\SpecialCharTok{+}
    \FunctionTok{ylab}\NormalTok{(}\StringTok{"Count"}\NormalTok{)}\SpecialCharTok{+}
    \FunctionTok{xlab}\NormalTok{(}\StringTok{"Latitude"}\NormalTok{)}
  
\NormalTok{gg.graphbylat }\SpecialCharTok{+} \FunctionTok{facet\_grid}\NormalTok{(id }\SpecialCharTok{\textasciitilde{}}\NormalTok{. , }\AttributeTok{scales =} \StringTok{"free"}\NormalTok{)}
\end{Highlighting}
\end{Shaded}

\begin{verbatim}
## `geom_smooth()` using formula = 'y ~ x'
\end{verbatim}

\includegraphics{C:/Users/sdwor/OneDrive - University of Guelph/1- MASTERS DEGREE/Software tools/assignment1_files/figure-latex/unnamed-chunk-1-3.pdf}

\begin{Shaded}
\begin{Highlighting}[]
\CommentTok{\#Making two species accumulation curve, one for above 40 lat, and one between 0 and 40 lat with iNEXT to incorporate extrapolation }
\CommentTok{\#Creating a new data set for only between 40 degrees latitude and the equator}

\NormalTok{df.nemsouth }\OtherTok{\textless{}{-}}\NormalTok{ df.Onchocercidae }\SpecialCharTok{\%\textgreater{}\%}
  \FunctionTok{filter}\NormalTok{(lat }\SpecialCharTok{\textless{}=}\DecValTok{40}\NormalTok{, lat }\SpecialCharTok{\textgreater{}=}\DecValTok{0}\NormalTok{)}

\CommentTok{\#creating a new data set for only above 40 degrees latitude}

\NormalTok{df.nemNorth }\OtherTok{\textless{}{-}}\NormalTok{ df.Onchocercidae }\SpecialCharTok{\%\textgreater{}\%}
  \FunctionTok{filter}\NormalTok{(lat }\SpecialCharTok{\textgreater{}=} \DecValTok{40}\NormalTok{)}

\CommentTok{\#Grouping per species and counting the number of each species in both the north and south subset }

\NormalTok{dfnemsouth.by.species }\OtherTok{\textless{}{-}}\NormalTok{ df.nemsouth }\SpecialCharTok{\%\textgreater{}\%}
  \FunctionTok{group\_by}\NormalTok{(species\_name) }\SpecialCharTok{\%\textgreater{}\%}
  \FunctionTok{count}\NormalTok{(species\_name)}

\FunctionTok{colnames}\NormalTok{(dfnemsouth.by.species)[}\DecValTok{2}\NormalTok{] }\OtherTok{\textless{}{-}} \StringTok{"num\_species\_south"}

\NormalTok{dfnemNorth.by.species }\OtherTok{\textless{}{-}}\NormalTok{ df.nemNorth }\SpecialCharTok{\%\textgreater{}\%}
  \FunctionTok{group\_by}\NormalTok{(species\_name) }\SpecialCharTok{\%\textgreater{}\%}
  \FunctionTok{count}\NormalTok{(species\_name)}

\FunctionTok{colnames}\NormalTok{(dfnemNorth.by.species)[}\DecValTok{2}\NormalTok{] }\OtherTok{\textless{}{-}} \StringTok{"num\_species\_north"}

\CommentTok{\#Converting the number of species data into one list}

\NormalTok{num\_species\_northd }\OtherTok{\textless{}{-}} \FunctionTok{as.double}\NormalTok{(}\FunctionTok{as.character}\NormalTok{(dfnemNorth.by.species}\SpecialCharTok{$}\NormalTok{num\_species\_north))}

\NormalTok{num\_species\_southd }\OtherTok{\textless{}{-}} \FunctionTok{as.double}\NormalTok{(}\FunctionTok{as.character}\NormalTok{(dfnemsouth.by.species}\SpecialCharTok{$}\NormalTok{num\_species\_south))}

\NormalTok{listnem }\OtherTok{\textless{}{-}} \FunctionTok{list}\NormalTok{(num\_species\_northd, num\_species\_southd)}

\CommentTok{\#Graph for predicting the total number of species in each latitudinal region with increased sampling }

\NormalTok{out }\OtherTok{\textless{}{-}} \FunctionTok{iNEXT}\NormalTok{(listnem, }\AttributeTok{q=}\DecValTok{0}\NormalTok{, }\AttributeTok{datatype=}\StringTok{"abundance"}\NormalTok{, }\AttributeTok{endpoint=}\DecValTok{150}\NormalTok{)}

\FunctionTok{ggiNEXT}\NormalTok{(out)}\SpecialCharTok{+}
  \FunctionTok{theme\_bw}\NormalTok{(}\AttributeTok{base\_size =} \DecValTok{15}\NormalTok{)}\SpecialCharTok{+}
  \FunctionTok{scale\_color\_manual}\NormalTok{(}\AttributeTok{labels =} \FunctionTok{c}\NormalTok{(}\StringTok{"North of 40"}\NormalTok{, }\StringTok{"South of 40"}\NormalTok{), }\AttributeTok{values =} \FunctionTok{c}\NormalTok{(}\StringTok{"\#D55E00"}\NormalTok{,}\StringTok{"\#0072B2"}\NormalTok{))}
\end{Highlighting}
\end{Shaded}

\begin{verbatim}
## Scale for colour is already present.
## Adding another scale for colour, which will replace the existing scale.
\end{verbatim}

\includegraphics{C:/Users/sdwor/OneDrive - University of Guelph/1- MASTERS DEGREE/Software tools/assignment1_files/figure-latex/unnamed-chunk-1-4.pdf}

\begin{Shaded}
\begin{Highlighting}[]
\CommentTok{\#Creating a bar graph for the species richness of the species north of 40 degree latitude }

\FunctionTok{ggplot}\NormalTok{(dfnemNorth.by.species, }\FunctionTok{aes}\NormalTok{(}\AttributeTok{x=}\NormalTok{num\_species\_north, }\AttributeTok{y=}\NormalTok{species\_name))}\SpecialCharTok{+}
  \FunctionTok{geom\_bar}\NormalTok{(}\AttributeTok{stat=} \StringTok{\textquotesingle{}identity\textquotesingle{}}\NormalTok{, }\AttributeTok{fill=}\FunctionTok{c}\NormalTok{(}\StringTok{"\#0072B2"}\NormalTok{))}\SpecialCharTok{+}
  \FunctionTok{theme\_bw}\NormalTok{(}\AttributeTok{base\_size =} \DecValTok{15}\NormalTok{)}\SpecialCharTok{+}
  \FunctionTok{theme}\NormalTok{(}\AttributeTok{axis.text.x =} \FunctionTok{element\_text}\NormalTok{(}\AttributeTok{size =} \DecValTok{10}\NormalTok{), }\AttributeTok{axis.text.y =} \FunctionTok{element\_text}\NormalTok{(}\AttributeTok{size =} \DecValTok{10}\NormalTok{), }\AttributeTok{axis.title =} \FunctionTok{element\_text}\NormalTok{(}\AttributeTok{margin =} \FunctionTok{margin}\NormalTok{(}\AttributeTok{t =} \DecValTok{0}\NormalTok{, }\AttributeTok{r =} \DecValTok{10}\NormalTok{, }\AttributeTok{b =} \DecValTok{10}\NormalTok{, }\AttributeTok{l =} \DecValTok{0}\NormalTok{), }\AttributeTok{size =} \DecValTok{12}\NormalTok{), }\AttributeTok{legend.title =} \FunctionTok{element\_text}\NormalTok{(}\AttributeTok{size =} \DecValTok{12}\NormalTok{), }\AttributeTok{legend.text =} \FunctionTok{element\_text}\NormalTok{(}\AttributeTok{size =} \DecValTok{10}\NormalTok{))}\SpecialCharTok{+}
  \FunctionTok{ylab}\NormalTok{(}\StringTok{"Species richness"}\NormalTok{)}\SpecialCharTok{+}
  \FunctionTok{xlab}\NormalTok{(}\StringTok{"Species"}\NormalTok{)}
\end{Highlighting}
\end{Shaded}

\includegraphics{C:/Users/sdwor/OneDrive - University of Guelph/1- MASTERS DEGREE/Software tools/assignment1_files/figure-latex/unnamed-chunk-1-5.pdf}

\end{document}
